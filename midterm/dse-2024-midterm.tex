
\documentclass[12pt]{article}

% \usepackage{physics}

\usepackage{hyperref}
\hypersetup{
    colorlinks=true,
    linkcolor=blue,
    filecolor=magenta,      
    urlcolor=cyan,
    pdftitle={Overleaf Example},
    pdfpagemode=FullScreen,
    }

\usepackage{tikzducks}

\usepackage{tikz} % картинки в tikz
\usepackage{microtype} % свешивание пунктуации

\usepackage{array} % для столбцов фиксированной ширины

\usepackage{indentfirst} % отступ в первом параграфе

\usepackage{sectsty} % для центрирования названий частей
\allsectionsfont{\centering}

\usepackage{amsmath, amsfonts, amssymb} % куча стандартных математических плюшек

\usepackage{comment}

\usepackage[top=2cm, left=1.2cm, right=1.2cm, bottom=2cm]{geometry} % размер текста на странице

\usepackage{lastpage} % чтобы узнать номер последней страницы

\usepackage{enumitem} % дополнительные плюшки для списков
%  например \begin{enumerate}[resume] позволяет продолжить нумерацию в новом списке
\usepackage{caption}

\usepackage{url} % to use \url{link to web}


\newcommand{\smallduck}{\begin{tikzpicture}[scale=0.3]
    \duck[
        cape=black,
        hat=black,
        mask=black
    ]
    \end{tikzpicture}}

\usepackage{fancyhdr} % весёлые колонтитулы
\pagestyle{fancy}
\lhead{Fall 2024}
\chead{Midterm}
\rhead{Data Science for Economists}
\lfoot{}
\cfoot{}
\rfoot{}

\renewcommand{\headrulewidth}{0.4pt}
\renewcommand{\footrulewidth}{0.4pt}

\usepackage{tcolorbox} % рамочки!

\usepackage{todonotes} % для вставки в документ заметок о том, что осталось сделать
% \todo{Здесь надо коэффициенты исправить}
% \missingfigure{Здесь будет Последний день Помпеи}
% \listoftodos - печатает все поставленные \todo'шки


% более красивые таблицы
\usepackage{booktabs}
% заповеди из докупентации:
% 1. Не используйте вертикальные линни
% 2. Не используйте двойные линии
% 3. Единицы измерения - в шапку таблицы
% 4. Не сокращайте .1 вместо 0.1
% 5. Повторяющееся значение повторяйте, а не говорите "то же"


\setcounter{MaxMatrixCols}{20}
% by crazy default pmatrix supports only 10 cols :)


\usepackage{fontspec}
\usepackage{libertine}
\usepackage{polyglossia}

\setmainlanguage{russian}
\setotherlanguages{english}

% download "Linux Libertine" fonts:
% http://www.linuxlibertine.org/index.php?id=91&L=1
% \setmainfont{Linux Libertine O} % or Helvetica, Arial, Cambria
% why do we need \newfontfamily:
% http://tex.stackexchange.com/questions/91507/
% \newfontfamily{\cyrillicfonttt}{Linux Libertine O}

\AddEnumerateCounter{\asbuk}{\russian@alph}{щ} % для списков с русскими буквами
% \setlist[enumerate, 2]{label=\asbuk*),ref=\asbuk*}

%% эконометрические сокращения
\DeclareMathOperator{\pCorr}{\mathrm{pCorr}}
\DeclareMathOperator{\Cov}{\mathbb{C}ov}
\DeclareMathOperator{\Corr}{\mathbb{C}orr}
\DeclareMathOperator{\Var}{\mathbb{V}ar}
\DeclareMathOperator{\col}{col}
\DeclareMathOperator{\row}{row}

\let\P\relax
\DeclareMathOperator{\P}{\mathbb{P}}

\let\H\relax
\DeclareMathOperator{\H}{\mathbb{H}}


\DeclareMathOperator{\E}{\mathbb{E}}
% \DeclareMathOperator{\tr}{trace}
\DeclareMathOperator{\card}{card}

\DeclareMathOperator{\Convex}{Convex}

\newcommand \cN{\mathcal{N}}
\newcommand \dN{\mathcal{N}}
\newcommand \dBin{\mathrm{Bin}}


\newcommand \RR{\mathbb{R}}
\newcommand \NN{\mathbb{N}}





\begin{document}

% \begin{enumerate}
%     \item entropy
%     \item horse betting
%     \item bootstrap 
%     \item regression tree
%     \item classification tree / logistic function 
%     \item random forest/gradient boosting
% \end{enumerate}

\begin{enumerate}

\item {[10]} The random variable $Y$ takes values $0$ or $1$ with $\P(Y = 1) = p$.
\begin{enumerate}
    \item {[3]} Find the entropy of $Y$ and plot it as a function of $p$.
    \item {[3]} Find the Gini impurity index of $Y$ and plot it as a function of $p$.
    \item {[3]} Find the second order Taylor expansion of the entropy as a function of $p$.
    \item {[1]} Is the second order entropy Taylor expansion exactly equal to the Gini impurity index?
\end{enumerate}
    

\item {[10]} Random variables $X$ and $Y$ are discrete and independent.

\begin{enumerate}
    \item {[3]} Is it possible that $\H(X + Y) = \H(X) + \H(Y)$? 
    Provide an example or prove that it is not possible. 
    \item {[3]} Is it possible that $\H(X + Y) < \H(X) + \H(Y)$? 
    Provide an example or prove that it is not possible. 
    \item {[4]} Is it possible that $\H(X + Y) > \H(X) + \H(Y)$? 
    Provide an example or prove that it is not possible. 
\end{enumerate}

\item {[10]} Each day you can bet any amount from zero up to your wealth.
Your bet is multiplied by $0.5$ or by $4$ with equal probabilities. 
You optimize your wealth in the long-run period. 

\begin{enumerate}
    \item {[8]} What is your optimal strategy?
    \item {[2]} Which maximal long-term daily interest rate is attainable?
\end{enumerate}


\item {[10]} Consider the following dataset $y = (1, 1, 1, 1, 0, 0)$, $a = (0, 0, 0, 1, 0, 1)$ and $b=(1, 1, 0, 0, 0, 1)$.
Here $y$ is the target variable and $a$ and $b$ are predictors. 

\begin{enumerate}
    \item {[8]} Construct a classification tree. 
    Use Gini impurity index as splitting criterion. 
    Grow the tree until it is not possible to split further. 
    \item {[2]} Calculate the total impurity drop due to each predictor. 
\end{enumerate}



\item {[10]} Elon Musk observes the random variable $x$ and forecasts the random variable $y$ using the formula $\hat y = 2 x$, but actually $y = 2 x^2 + u$.
Random variables $x$ and $u$ are independent and uniform on $[0;1]$.

\begin{enumerate}
    \item {[2]} Calculate the mean squared error of the forecast, $MSE = \E((y - \hat y)^2 )$.
    \item {[8]} Decompose the mean squared error onto three parts: part due to forecast bias, part due to forecast variance and unpredictable part. 
\end{enumerate}


\item {[10]} Random variables $y_1$, \dots, $y_n$ are independent identically distributed with $\P(y_i = 1) = p$, $\P(y_i = 0) = 1 - p$. 
Consider a naive bootstrap sample $y_1^*$, \dots, $y_n^*$.
\begin{enumerate}
    \item {[4]} Find $\P(y_1^* = y_1)$ and $\P(y_1^* = y_2^*)$.
    \item {[4]} Find $\E(y_i^*)$ and $\Var(y_i^*)$.
    \item {[2]} Find $\Cov(y_1^*, y_2^*)$.
\end{enumerate}


\end{enumerate}


\end{document}
